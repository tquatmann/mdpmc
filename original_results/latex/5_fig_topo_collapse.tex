
\begin{figure}[t]
%		\setlength{\quantileplotwidth}{0.7\textwidth}
%		\setlength{\quantileplotheight}{5cm}
	\centering
	{
		\quantileplot{plotdata/quantile-claix23-community.csv}
		{logs.Storm.vi-mono/color1, logs.Storm.vi-topo/color2, logs.Storm.vi-mono-mecq/color3, logs.Storm.vi-topo-mecq/color4}
		{VI, VI-topo, VI-mec, VI-mec-topo}
		{0}{\numcommunity}{0.1}{900}{north west}%
	}

	{
		\quantileplot{plotdata/quantile-claix23-community.csv}
		{logs.Storm.pi-mono-gmres/color1, logs.Storm.pi-topo-gmres/color2}%, logs.Storm.pi-mono-mecq/color3, logs.Storm.pi-mecq-topo/color4}
		{PI, PI-topo}%, PI-mec, PI-mec-topo}
		{0}{\numcommunity}{0.1}{900}{north west}%
	}

	{
		\quantileplot{plotdata/quantile-claix23-community.csv}
		{logs.Storm.lp-mono-gurobi-4auto/color1, logs.Storm.lp-topo-gurobi-4auto/color2, logs.Storm.lp-mono-mecq-gurobi-4auto/color3, logs.Storm.lp-topo-mecq-gurobi-4auto/color4}
		{LP, LP-topo, LP-mec, LP-mec-topo}
		{0}{\numcommunity}{0.1}{900}{north west}%
	}
	\caption{Comparison of the vanilla algorithms VI, PI and LP and their variants using MEC collapsing and topological decomposition.}
	\label{fig:qvbs-topo-coll}
\end{figure}
\todo{PI with mec quotient is apparently not implemented}
